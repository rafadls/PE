\documentclass[]{article}
\usepackage{lmodern}
\usepackage{amssymb,amsmath}
\usepackage{ifxetex,ifluatex}
\usepackage{fixltx2e} % provides \textsubscript
\ifnum 0\ifxetex 1\fi\ifluatex 1\fi=0 % if pdftex
  \usepackage[T1]{fontenc}
  \usepackage[utf8]{inputenc}
\else % if luatex or xelatex
  \ifxetex
    \usepackage{mathspec}
  \else
    \usepackage{fontspec}
  \fi
  \defaultfontfeatures{Ligatures=TeX,Scale=MatchLowercase}
\fi
% use upquote if available, for straight quotes in verbatim environments
\IfFileExists{upquote.sty}{\usepackage{upquote}}{}
% use microtype if available
\IfFileExists{microtype.sty}{%
\usepackage{microtype}
\UseMicrotypeSet[protrusion]{basicmath} % disable protrusion for tt fonts
}{}
\usepackage[margin=1in]{geometry}
\usepackage{hyperref}
\hypersetup{unicode=true,
            pdftitle={Tarea 1},
            pdfborder={0 0 0},
            breaklinks=true}
\urlstyle{same}  % don't use monospace font for urls
\usepackage{color}
\usepackage{fancyvrb}
\newcommand{\VerbBar}{|}
\newcommand{\VERB}{\Verb[commandchars=\\\{\}]}
\DefineVerbatimEnvironment{Highlighting}{Verbatim}{commandchars=\\\{\}}
% Add ',fontsize=\small' for more characters per line
\usepackage{framed}
\definecolor{shadecolor}{RGB}{248,248,248}
\newenvironment{Shaded}{\begin{snugshade}}{\end{snugshade}}
\newcommand{\AlertTok}[1]{\textcolor[rgb]{0.94,0.16,0.16}{#1}}
\newcommand{\AnnotationTok}[1]{\textcolor[rgb]{0.56,0.35,0.01}{\textbf{\textit{#1}}}}
\newcommand{\AttributeTok}[1]{\textcolor[rgb]{0.77,0.63,0.00}{#1}}
\newcommand{\BaseNTok}[1]{\textcolor[rgb]{0.00,0.00,0.81}{#1}}
\newcommand{\BuiltInTok}[1]{#1}
\newcommand{\CharTok}[1]{\textcolor[rgb]{0.31,0.60,0.02}{#1}}
\newcommand{\CommentTok}[1]{\textcolor[rgb]{0.56,0.35,0.01}{\textit{#1}}}
\newcommand{\CommentVarTok}[1]{\textcolor[rgb]{0.56,0.35,0.01}{\textbf{\textit{#1}}}}
\newcommand{\ConstantTok}[1]{\textcolor[rgb]{0.00,0.00,0.00}{#1}}
\newcommand{\ControlFlowTok}[1]{\textcolor[rgb]{0.13,0.29,0.53}{\textbf{#1}}}
\newcommand{\DataTypeTok}[1]{\textcolor[rgb]{0.13,0.29,0.53}{#1}}
\newcommand{\DecValTok}[1]{\textcolor[rgb]{0.00,0.00,0.81}{#1}}
\newcommand{\DocumentationTok}[1]{\textcolor[rgb]{0.56,0.35,0.01}{\textbf{\textit{#1}}}}
\newcommand{\ErrorTok}[1]{\textcolor[rgb]{0.64,0.00,0.00}{\textbf{#1}}}
\newcommand{\ExtensionTok}[1]{#1}
\newcommand{\FloatTok}[1]{\textcolor[rgb]{0.00,0.00,0.81}{#1}}
\newcommand{\FunctionTok}[1]{\textcolor[rgb]{0.00,0.00,0.00}{#1}}
\newcommand{\ImportTok}[1]{#1}
\newcommand{\InformationTok}[1]{\textcolor[rgb]{0.56,0.35,0.01}{\textbf{\textit{#1}}}}
\newcommand{\KeywordTok}[1]{\textcolor[rgb]{0.13,0.29,0.53}{\textbf{#1}}}
\newcommand{\NormalTok}[1]{#1}
\newcommand{\OperatorTok}[1]{\textcolor[rgb]{0.81,0.36,0.00}{\textbf{#1}}}
\newcommand{\OtherTok}[1]{\textcolor[rgb]{0.56,0.35,0.01}{#1}}
\newcommand{\PreprocessorTok}[1]{\textcolor[rgb]{0.56,0.35,0.01}{\textit{#1}}}
\newcommand{\RegionMarkerTok}[1]{#1}
\newcommand{\SpecialCharTok}[1]{\textcolor[rgb]{0.00,0.00,0.00}{#1}}
\newcommand{\SpecialStringTok}[1]{\textcolor[rgb]{0.31,0.60,0.02}{#1}}
\newcommand{\StringTok}[1]{\textcolor[rgb]{0.31,0.60,0.02}{#1}}
\newcommand{\VariableTok}[1]{\textcolor[rgb]{0.00,0.00,0.00}{#1}}
\newcommand{\VerbatimStringTok}[1]{\textcolor[rgb]{0.31,0.60,0.02}{#1}}
\newcommand{\WarningTok}[1]{\textcolor[rgb]{0.56,0.35,0.01}{\textbf{\textit{#1}}}}
\usepackage{graphicx,grffile}
\makeatletter
\def\maxwidth{\ifdim\Gin@nat@width>\linewidth\linewidth\else\Gin@nat@width\fi}
\def\maxheight{\ifdim\Gin@nat@height>\textheight\textheight\else\Gin@nat@height\fi}
\makeatother
% Scale images if necessary, so that they will not overflow the page
% margins by default, and it is still possible to overwrite the defaults
% using explicit options in \includegraphics[width, height, ...]{}
\setkeys{Gin}{width=\maxwidth,height=\maxheight,keepaspectratio}
\IfFileExists{parskip.sty}{%
\usepackage{parskip}
}{% else
\setlength{\parindent}{0pt}
\setlength{\parskip}{6pt plus 2pt minus 1pt}
}
\setlength{\emergencystretch}{3em}  % prevent overfull lines
\providecommand{\tightlist}{%
  \setlength{\itemsep}{0pt}\setlength{\parskip}{0pt}}
\setcounter{secnumdepth}{0}
% Redefines (sub)paragraphs to behave more like sections
\ifx\paragraph\undefined\else
\let\oldparagraph\paragraph
\renewcommand{\paragraph}[1]{\oldparagraph{#1}\mbox{}}
\fi
\ifx\subparagraph\undefined\else
\let\oldsubparagraph\subparagraph
\renewcommand{\subparagraph}[1]{\oldsubparagraph{#1}\mbox{}}
\fi

%%% Use protect on footnotes to avoid problems with footnotes in titles
\let\rmarkdownfootnote\footnote%
\def\footnote{\protect\rmarkdownfootnote}

%%% Change title format to be more compact
\usepackage{titling}

% Create subtitle command for use in maketitle
\providecommand{\subtitle}[1]{
  \posttitle{
    \begin{center}\large#1\end{center}
    }
}

\setlength{\droptitle}{-2em}

  \title{Tarea 1}
    \pretitle{\vspace{\droptitle}\centering\huge}
  \posttitle{\par}
    \author{}
    \preauthor{}\postauthor{}
    \date{}
    \predate{}\postdate{}
  

\begin{document}
\maketitle

\includegraphics{banner.png}

Tarea 1: Foundations

CC6104: Statistical Thinking

\hypertarget{integrantes}{%
\paragraph{\texorpdfstring{\textbf{Integrantes
:}}{Integrantes :}}\label{integrantes}}

\begin{itemize}
\tightlist
\item
  Rafael De La Sotta
\item
  Felipe Ortuzar
\end{itemize}

\hypertarget{cuerpo-docente}{%
\paragraph{\texorpdfstring{\textbf{Cuerpo
Docente:}}{Cuerpo Docente:}}\label{cuerpo-docente}}

\begin{itemize}
\tightlist
\item
  Profesor: Felipe Bravo M.
\item
  Auxiliar: Sebastian Bustos e Ignacio Meza D.
\end{itemize}

\hypertarget{fecha-limite-de-entrega}{%
\paragraph{\texorpdfstring{\textbf{Fecha límite de
entrega:}}{Fecha límite de entrega:}}\label{fecha-limite-de-entrega}}

\hypertarget{indice}{%
\subsubsection{\texorpdfstring{\textbf{Índice:}}{Índice:}}\label{indice}}

\begin{enumerate}
\def\labelenumi{\arabic{enumi}.}
\tightlist
\item
  \protect\hyperlink{id1}{Objetivo}
\item
  \protect\hyperlink{id2}{Instrucciones}
\item
  \protect\hyperlink{id3}{Referencias}
\item
  \protect\hyperlink{id4}{Primera Parte: Preguntas Teóricas}
\item
  \protect\hyperlink{id5}{Segunda Parte: Elaboración de Código}
\end{enumerate}

\hypertarget{objetivo}{%
\subsubsection{\texorpdfstring{\textbf{Objetivo}}{Objetivo}}\label{objetivo}}

\href{mailto:Bienvenid@s}{\nolinkurl{Bienvenid@s}} a la primera tarea
del curso Statistical Thinking. Esta tarea tiene como objetivo evaluar
los contenidos teóricos de la primera parte del curso, los cuales se
enfocan principalmente en análisis exploratorio de datos y conceptos
introductorios de probabilidades. Si aún no han visto las clases, se
recomienda visitar los enlaces de las referencias.

La tarea consta de una parte teórica que busca evaluar conceptos vistos
en clases. Seguido por una parte práctica con el fín de introducirlos a
la programación en R enfocada en el análisis estadístico de datos.

\hypertarget{instrucciones}{%
\subsubsection{\texorpdfstring{\textbf{Instrucciones:}}{Instrucciones:}}\label{instrucciones}}

\begin{itemize}
\tightlist
\item
  La tarea se realiza en grupos de \textbf{máximo 2 personas}. Pero no
  existe problema si usted desea hacerla de forma individual.
\item
  La entrega es a través de u-cursos a más tardar el día estipulado en
  la misma plataforma. A las tareas atrasadas se les descontará un punto
  por día.
\item
  El formato de entrega es este mismo \textbf{Rmarkdown} y un
  \textbf{html} con la tarea desarrollada. Por favor compruebe que todas
  las celdas han sido ejecutadas en el archivo html.
\item
  Al momento de la revisión tu código será ejecutado. Por favor verifica
  que tu entrega no tenga errores de compilación.
\item
  No serán revisadas tareas desarrolladas en Python.
\item
  Está \textbf{PROHIBIDO} la copia o compartir las respuestas entre
  integrantes de diferentes grupos.
\item
  Pueden realizar consultas de la tarea a través de U-cursos y/o del
  canal de Discord del curso.
\end{itemize}

\hypertarget{referencias}{%
\subsubsection{\texorpdfstring{\textbf{Referencias:}}{Referencias:}}\label{referencias}}

Slides de las clases:

\begin{itemize}
\tightlist
\item
  \href{https://github.com/dccuchile/CC6104/blob/master/slides/1_1_ST-intro.pdf}{Introduction
  to Statistical Thinking}
\item
  \href{https://github.com/dccuchile/CC6104/blob/master/slides/1_2_ST-R.pdf}{Introduction
  to R}
\item
  \href{https://github.com/dccuchile/CC6104/blob/master/slides/1_3_ST-explore.pdf}{Descriptive
  Statistics}
\item
  \href{https://github.com/dccuchile/CC6104/blob/master/slides/1_4_ST-prob.pdf}{Probability}
\end{itemize}

Videos de las clases:

\begin{itemize}
\tightlist
\item
  Introduction to Statistical Thinking:
  \href{https://youtu.be/X4SqJu6lExM}{video1}
  \href{https://youtu.be/YbiQU5TTBX4}{video2}
\item
  Introduction to R: \href{https://youtu.be/MbeLD3hWWVo}{video1}
  \href{https://youtu.be/9W_eWCy86F4}{video2}
  \href{https://youtu.be/QvFXSw2-1r4}{video3}
  \href{https://youtu.be/y4JY7klrbfQ}{video4}
\item
  Descriptive Statistics: \href{https://youtu.be/kWNskZ8_98o}{video1}
  \href{https://youtu.be/_FJ8x9M4b1w}{video2}
  \href{https://youtu.be/m7VBNZ2mYWI}{video3}
  \href{https://youtu.be/ylGMJ_aSQk0}{video4}
\item
  Probability: \href{https://youtu.be/R9AVYV73m1M}{video1}
  \href{https://youtu.be/zubh1jbRiKE}{video2}
  \href{https://youtu.be/uiwToagp0z4}{video3}
  \href{https://youtu.be/RlhN3t_VIyw}{video4}
  \href{https://youtu.be/4kV1dBaeWVc}{video5}
  \href{https://youtu.be/MGyXc70JdSk}{video6}
\end{itemize}

\hypertarget{primera-parte-preguntas-teoricas}{%
\section{Primera Parte: Preguntas
Teóricas}\label{primera-parte-preguntas-teoricas}}

A continuación, se presentaran diferentes preguntas que abordan las
temáticas vistas en clases. Por favor responda cada una de estas
preguntas de forma breve, no más de 4 o 5 lineas.

\hypertarget{pregunta-1}{%
\paragraph{\texorpdfstring{\textbf{Pregunta
1:}}{Pregunta 1:}}\label{pregunta-1}}

¿Por qué la estadística es importante?, ¿Que nos permite realizar con
los datos?. De algún ejemplo.

La estadística tiene como tema central la organización y análisis de
datos. Con ella se puede describir, decidir y predecir a partir de
estos. La estadística logra transformar los datos en información
confiable sobre fenomenos, y esta información, debido a su
confiabilidad, permite predecir, detectar tendencias y tomar desiciones.

\hypertarget{pregunta-2}{%
\paragraph{\texorpdfstring{\textbf{Pregunta
2:}}{Pregunta 2:}}\label{pregunta-2}}

Un amigue cercano a usted le comenta que le preocupa salir a la calle
cuando hay ofertas en los helados, esto debido a que ha visto el
siguiente titular en un famoso diario chileno: ``El aumento en la compra
de helados tiene una alta correlación con la muerte de personas en
Santiago''. ¿Que le recomendaría a su amigue sobre el titular leído?,
¿Debería preocuparse tanto?.

\begin{quote}
Respuesta Aquí
\end{quote}

\hypertarget{pregunta-3}{%
\paragraph{\texorpdfstring{\textbf{Pregunta
3:}}{Pregunta 3:}}\label{pregunta-3}}

Señale las diferentes aplicaciones que poseen las visualizaciones:
Boxplot, histograma, gráfico de pie y scatterplot.

El Boxplot está hecho en base a los percentiles. Nos permite identificar
la distribución en base a sus cuartiles, su simetría y sus outliers. A
continuación se muestras sus aplicaciónes: * Boxplot de solo una
variable * Separa una variable en categorias y hacer boxplot de cada una
de estas. * Comparar distintos boxplot en un mismo gráfico.

El histograma muestra la distribución de valores de una variable. Se
aplica para descomponer variables con un cierto orden, pues los datos
son agrupados en ``bins'', permitiendo elegír la continuidad del
gráfico. A continuación se muestras sus aplicaciónes: * Histograma por
cantidad * Histograma por densidad (áreas deben sumar 1)

El gráfico de pie es utilizado para mostrar frecuencia de clases en una
variable. Normalmente se utiliza para variables categóricas. A
continuación se muestras sus aplicaciónes: * Dos dimensiones * Tres
dimensiones

Scatterplot compara dos variables numéricas en un plano carteciano,
donde los valores de cada dato determinan su posición. A continuación se
muestras sus aplicaciónes: * Scatterplot entre dos variables numericas.
* Combinación de scatterplot entre un grupo de variables. * Scatterplot
de tres dimensiones.

\hypertarget{pregunta-4}{%
\paragraph{\texorpdfstring{\textbf{Pregunta
4:}}{Pregunta 4:}}\label{pregunta-4}}

Suponga que esta estudiando la diferencia en los sueldos de las personas
que viven en Santiago y Rancagua. Suponiendo que los datos poseen
outliers, ¿Que métrica de resumen utilizaría para comparar los datos?.
Justifique su respuesta.

\begin{quote}
Respuesta Aquí
\end{quote}

\hypertarget{pregunta-5}{%
\paragraph{\texorpdfstring{\textbf{Pregunta
5:}}{Pregunta 5:}}\label{pregunta-5}}

En base al mismo dataset de sueldos para las regiones de Santiago y
Rancagua, le comentan que existe un error en los datos y que estos deben
ser modificados aumentando un 10\% el valor original y sumando
\(15.000\) a cada uno de los datos. ¿Como se ve afectada la media,
mediana y desviación estándar con esta modificación?. Explique a través
de ecuaciones el cambio que experimentan las métricas de resumen
respecto al valor original, considere para el caso de la media
\(\bar{X}_{old} = \dfrac{1}{m} \sum^{m}_{i=1} x_i\) y
\(sd_{old} = \sqrt{\dfrac{1}{(m-1)}\sum_{i=1}^{m}(x_i-\bar{x})^{2}}\)
para la desviación estándar.

\begin{quote}
Respuesta Aquí
\end{quote}

\hypertarget{pregunta-6}{%
\paragraph{\texorpdfstring{\textbf{Pregunta
6:}}{Pregunta 6:}}\label{pregunta-6}}

Suponga que debe responder un examen sorpresa de 10 preguntas, con 5
alternativas por cada pregunta. ¿Cual es la probabilidad de obtener mas
de 5 alternativas correctas si responde de forma aleatoria todo el
examen?.

\textbf{Nota:} Puede resolver el ejercicio desarrollándolo a mano o
utilizando código en R.

\begin{quote}
Respuesta Aquí
\end{quote}

\hypertarget{pregunta-7}{%
\paragraph{\texorpdfstring{\textbf{Pregunta
7:}}{Pregunta 7:}}\label{pregunta-7}}

Supongamos que el 10\% de los alumnos del curso utilizan Macintosh, el
60\% utiliza Windows y el 30\% utiliza Linux. Supongamos que el 50\% de
los usuarios de Mac, el 78\% de los usuarios de Windows y el 20\% de los
usuarios de Linux han sucumbido bajo un terrible virus. Al seleccionar
una persona al azar nos enteramos de que su sistema está infectado por
el virus. ¿Cuál es la probabilidad de que sea un alumno con Windows?.

\begin{quote}
Respuesta Aquí
\end{quote}

\hypertarget{pregunta-8}{%
\paragraph{\texorpdfstring{\textbf{Pregunta
8:}}{Pregunta 8:}}\label{pregunta-8}}

Señale si las siguientes declaraciones son verdaderas o falsas respecto
a las variables aleatorias:

\begin{itemize}
\tightlist
\item[$\square$]
  Como las variables aleatorias son funciones que nos permiten obtener
  valores de probabilidad, siempre podemos obtener \(\mathbb{P}(X=x)>0\)
  evaluando en una \(f(x)\) continua y discreta.
\item[$\square$]
  Una PDF bien definida solo puede tener valores menores a 1 y un área
  debajo de la curva igual a 1.
\item[$\square$]
  La CDF puede ser representada como la integral de la PDF y PMF.
\item[$\square$]
  Una CDF es definida para todo x, continua hacia la derecha y no es
  decreciente.
\end{itemize}

\begin{quote}
Respuesta Aquí
\end{quote}

\hypertarget{pregunta-9}{%
\paragraph{\texorpdfstring{\textbf{Pregunta
9:}}{Pregunta 9:}}\label{pregunta-9}}

Una famosa fabrica de dulces señala que solo el \(5\%\) de sus dulces
contienen menos de \(350\) gramos. Si los dulces elaborados por la
fabrica distribuyen de forma normal, con media \(\mu\) y desviación
estándar \(11.2\). Responda las siguientes preguntas:

\begin{itemize}
\item
  \begin{enumerate}
  \def\labelenumi{\alph{enumi})}
  \tightlist
  \item
    Encuentre la media del producto.
  \end{enumerate}
\item
  \begin{enumerate}
  \def\labelenumi{\alph{enumi})}
  \setcounter{enumi}{1}
  \tightlist
  \item
    Señale el porcentaje de dulces que se encuentran sobre los \(390\)
    gramos.
  \end{enumerate}
\end{itemize}

\textbf{Nota:} Puede ser útil
\url{https://www.statskingdom.com/z_table.html}

\begin{quote}
Respuesta Aquí
\end{quote}

\begin{center}\rule{0.5\linewidth}{0.5pt}\end{center}

\hypertarget{segunda-parte-elaboracion-de-codigo}{%
\section{Segunda Parte: Elaboración de
Código}\label{segunda-parte-elaboracion-de-codigo}}

En la siguiente sección deberá resolver cada uno de los experimentos
computacionales a través de la programación en R. Para esto se le
aconseja que cree funciones en R, ya que le facilitará la ejecución de
gran parte de lo solicitado.

\hypertarget{pregunta-1-visualizacion-de-datos}{%
\subsubsection{Pregunta 1: Visualización de
Datos}\label{pregunta-1-visualizacion-de-datos}}

Para esta pregunta usted deberá trabajar en base al conjunto de datos
\texttt{hearth\_database.csv}, el cual esta compuesto por las siguientes
variables:

\begin{itemize}
\tightlist
\item
  target: Señala si el paciente tuvo un infarto.
\item
  sex: Sexo de los sujetos de prueba.
\item
  fbs: Azúcar en la sangre con ayunas. Esta variable señala solo si se
  encuentra \textless=120 o \textgreater120.
\item
  exang: Angina de pecho inducida por el ejercicio.
\item
  cp: Tipo de dolor de pecho.
\item
  restecg: Resultados electrocardiográficos en reposo.
\item
  slope: Pendiente del segmento ST máximo de ejercicio.
\item
  ca: Número de buques principales.
\item
  thal: Thalassemia.
\item
  age: Edad en años.
\item
  trestbps: Presión arterial en reposo.
\item
  chol: colesterol sérico en mg/dl.
\item
  thalach: Frecuencia cardíaca máxima alcanzada.
\item
  oldpeak: Depresión del ST inducida por el ejercicio en relación con el
  reposo.
\end{itemize}

En base al dataset propuesto realice un análisis exploratorio de los
datos (EDA). Para su análisis enfoquen el desarrollo en las siguientes
tareas:

\begin{itemize}
\tightlist
\item[$\square$]
  Obtenga la media, mediana, quintiles y valores máximos desde los datos
  que componen el dataset.
\item[$\square$]
  Obtenga la Matriz de correlación de Pearson y visualice la relación
  entre las variables numéricas.
\item[$\square$]
  Visualice los boxplot para las variables numéricas.
\item[$\square$]
  Visualice a través de un histograma como distribuyen las variables
  respecto a los TARGET.
\end{itemize}

\textbf{Respuesta}

\begin{center}\rule{0.5\linewidth}{0.5pt}\end{center}

\hypertarget{pregunta-2-teorema-central-del-limite}{%
\subsubsection{Pregunta 2: Teorema Central del
Limite}\label{pregunta-2-teorema-central-del-limite}}

Pruebe el teorema central del limite aplicando un muestreo de la media
en las distribuciones Poisson, Exponencial y una a su elección. Grafique
los resultados obtenidos y señale aproximadamente el numero de muestreos
necesarios para obtener el resultado esperado, pruebe esto con las
siguientes cantidades de muestreo \(\{10,100,1000,5000\}\). ¿El efecto
ocurre con el mismo número de muestreo para todas las distribuciones?.

\textbf{Respuesta}

\begin{Shaded}
\begin{Highlighting}[]
\CommentTok{\# Definición de variables o estructuras necesarias para el muestreo.}
\NormalTok{n }\OtherTok{\textless{}{-}} \DecValTok{1000}

\CommentTok{\# Realizar el muestreo de las distribuciones.}
\ControlFlowTok{for}\NormalTok{(i }\ControlFlowTok{in} \DecValTok{1}\SpecialCharTok{:}\NormalTok{n) \{}
  
\NormalTok{\}}

\CommentTok{\# Código para gráficos}
\end{Highlighting}
\end{Shaded}

\begin{center}\rule{0.5\linewidth}{0.5pt}\end{center}

\hypertarget{pregunta-3-ley-de-los-grandes-numeros.}{%
\subsubsection{Pregunta 3: Ley de los Grandes
Numeros.}\label{pregunta-3-ley-de-los-grandes-numeros.}}

\hypertarget{lanzamiento-de-monedas}{%
\paragraph{Lanzamiento de monedas}\label{lanzamiento-de-monedas}}

Realice el experimento de lanzar una moneda cargada 1000 veces y observe
el comportamiento que tiene la probabilidad de salir cara. Para realizar
el experimento considere que la moneda tiene una probabilidad de \(4/5\)
de salir cara. Grafique el experimento para las secuencias de intentos
que van desde 1 a 1000, señalando el valor en que converge la
probabilidad de salir cara.

\textbf{Respuesta}

\begin{Shaded}
\begin{Highlighting}[]
\CommentTok{\# Simular lanzamientos}
\ControlFlowTok{for}\NormalTok{ (lanzamientos }\ControlFlowTok{in} \DecValTok{1}\SpecialCharTok{:}\DecValTok{1000}\NormalTok{) \{}
  
\NormalTok{\}}

\CommentTok{\# Gráfico de la convergencia}
\end{Highlighting}
\end{Shaded}

\hypertarget{el-problema-de-monty-hall}{%
\paragraph{El problema de Monty Hall}\label{el-problema-de-monty-hall}}

Remontándonos en la televisión del año 1963, en USA existía un programa
de concursos donde los participantes debían escoger entre 3 puertas para
ganar un premio soñado. El problema del concurso era que solo detrás de
1 puerta estaba el premio mayor, mientras que detrás de las otras dos
habían cabras como ``premio''.

Una de las particularidades de este concurso, es que cuando el
participante escogía una puerta, el animador del show abría una de las
puertas que no fue escogida por el participante (Obviamente la puerta
abierta por el animador no contenía el premio). Tras abrir la puerta, el
animador consultaba al participante si su elección era definitiva, o si
deseaba cambiar la puerta escogida por la otra puerta cerrada.

Imagine que usted es participante del concurso y desea calcular la
probabilidad de ganar el gran premio \textbf{si cambia de puerta} en el
momento que el animador se lo ofrece. Utilizando listas/arrays/vectores
simule las puertas del concurso, dejando aleatoriamente el premio en
alguna posición del array. Hecho esto, genere un numero de forma
aleatoria para escoger una de las puerta (posiciones de la estructura),
para luego ver si cambiando de posición tendrá mayores posibilidades de
ganar el premio. Genere N veces el experimento y grafique cada una de
las iteraciones, tal como se hizo en el ejercicio de las monedas.

\textbf{Respuesta:}

\begin{Shaded}
\begin{Highlighting}[]
\CommentTok{\# Creamos una función que simule el juego}
\NormalTok{montyhall }\OtherTok{\textless{}{-}} \ControlFlowTok{function}\NormalTok{(}\AttributeTok{cambiar =} \ConstantTok{TRUE}\NormalTok{)\{}
\NormalTok{  Puertas }\OtherTok{\textless{}{-}} \FunctionTok{sample}\NormalTok{(}\DecValTok{1}\SpecialCharTok{:}\DecValTok{3}\NormalTok{,}\DecValTok{3}\NormalTok{)             }\CommentTok{\#Puertas donde la posición que tiene el 3 es el premio}
\NormalTok{  posicion }\OtherTok{\textless{}{-}} \FunctionTok{sample}\NormalTok{(}\DecValTok{1}\SpecialCharTok{:}\DecValTok{3}\NormalTok{,}\DecValTok{1}\NormalTok{)            }\CommentTok{\#Elección del participante.}
  
  \FunctionTok{return}\NormalTok{(Eleccion) }\CommentTok{\# Retornamos la elección, esta puede que tenga el premio o no}
\NormalTok{\}}

\CommentTok{\# Función que simula N juegos}
\NormalTok{n\_juegos }\OtherTok{\textless{}{-}} \ControlFlowTok{function}\NormalTok{(}\AttributeTok{n =} \DecValTok{10}\NormalTok{ ,}\AttributeTok{cambiar\_puerta =} \ConstantTok{TRUE}\NormalTok{)\{}
  
\NormalTok{\}}
\end{Highlighting}
\end{Shaded}

\begin{center}\rule{0.5\linewidth}{0.5pt}\end{center}

\hypertarget{pregunta-4-independencia}{%
\subsubsection{Pregunta 4:
¿Independencia?}\label{pregunta-4-independencia}}

Ustedes disponen de los dados D1 y D2, los cuales no están cargados y
son utilizados para comprobar que
\(\mathbb{P}(AB)=\mathbb{P}(A)\mathbb{P}(B)\) cuando el evento A es
independiente del B. Para estudiar la independencia considere que los
eventos A y B se definen de la siguiente manera; sea A el evento dado
por los valores obtenidos en el lanzamiento del dado D1, este está
compuesto por \(A=\{D1=1,D1=2,D1=6\}\). Por otro lado, el evento B viene
dado por los valores obtenidos con el dado D2, el que está conformado
por \(B=\{D2=1,D2=2,D2=3,D2=4\}\). Con esto, tendremos un
\(\mathbb{P}(A)=1/2\), \(\mathbb{P}(𝐵)=2/3\) y \(\mathbb{P}(AB)=1/3\).
Compruebe de forma gráfica que al realizar 1000 lanzamientos (u otro
valor grande que usted desea probar) se visualiza que
\(\mathbb{P}(AB)=\mathbb{P}(A)\mathbb{P}(B)\).

Hecho lo anterior, compruebe el comportamiento de un segundo grupo de
eventos, dados por el lanzamiento de solo el dado D1. Donde, los eventos
para D1 quedan definidos como: \(A =\{D1=1,D1=2,D1=6\}\) y
\(B=\{D1=1,D1=2,D1=3\}\). ¿Se observa independencia en este
experimento?.

Se le aconseja que para simular los lanzamientos de dados utilice la
función \texttt{sample()} para generar valores aleatorios entre 1 y 6.
Compruebe los números generados por la función con los casos favorables
de cada uno de los eventos a ser estudiados.

\textbf{Respuesta:}

\begin{Shaded}
\begin{Highlighting}[]
\CommentTok{\# Primer grupo de eventos}
\NormalTok{N\_lan }\OtherTok{=} \DecValTok{1000} \CommentTok{\# Numero de lanzamientos}
  
\NormalTok{L\_A }\OtherTok{=}  \CommentTok{\# Lanzamientos favorables A = c(1, 2, 6)}
\NormalTok{L\_B }\OtherTok{=}  \CommentTok{\# Lanzamientos favorables B = c(1, 2, 3, 4)}
\NormalTok{L\_AB }\OtherTok{=} \CommentTok{\# Lanzamientos favorables AB = c(1, 2)}
\end{Highlighting}
\end{Shaded}

\begin{Shaded}
\begin{Highlighting}[]
\CommentTok{\# Segundo grupo de eventos}
\NormalTok{N\_lan }\OtherTok{=} \DecValTok{1000} \CommentTok{\# Numero de lanzamientos}
  
\NormalTok{L\_A }\OtherTok{=}  \CommentTok{\# Lanzamientos favorables A = c(1, 2, 6)}
\NormalTok{L\_B }\OtherTok{=}  \CommentTok{\# Lanzamientos favorables B = c(1, 2, 3)}
\NormalTok{L\_AB }\OtherTok{=} \CommentTok{\# Lanzamientos favorables AB = c(1, 2)}
\end{Highlighting}
\end{Shaded}

\begin{center}\rule{0.5\linewidth}{0.5pt}\end{center}

\hypertarget{pregunta-5-la-ruina-del-jugador}{%
\subsubsection{Pregunta 5: La Ruina del
Jugador}\label{pregunta-5-la-ruina-del-jugador}}

Un amigo ludópata suyo le comenta que el truco de jugar en el casino
esta en no parar de apostar y apostando lo mínimo posible. Ya que así,
tienes mas probabilidades de ganar el gran pozo que acumula el juego.
Usted sabiendo la condición de su amigo, decide no creer en su conjetura
y decide probar esto a través de un experimento.

Para realizar el experimento usted decide asumir las siguientes
declaraciones, bajo sus observaciones:

\begin{itemize}
\tightlist
\item
  La probabilidad de ganar en un juego del casino es \(9/19\)
\item
  Sabe que su amigo posee fondos en el rango de 0 a 200 dolares.
\item
  Las apuestas como mínimo deben ser igual a 5 dolares.
\item
  El monto de las apuestas no cambia y son siempre igual a la primera.
  Por ejemplo, si su amigo apuesta 50 dolares, todos los próximos juegos
  apuesta 50 hasta que se acaba su dinero.
\item
  Asuma que al momento de ganar el jugador anexa el valor apostado a sus
  fondos.
\end{itemize}

En el experimento deberá obtener la evolución de los fondos hasta que el
jugador se queda sin fondos para jugar. Puede ser útil seguir la lógica
de una moneda cargada para realizar esto. Pruebe esto con una apuesta
igual a 5, 25 y 50 graficando los resultados. Comente los resultados
obtenidos.

\textbf{Respuesta}

\begin{Shaded}
\begin{Highlighting}[]
\CommentTok{\# Función para obtener el desarrollo de las apuestas}
\NormalTok{ruina }\OtherTok{\textless{}{-}} \ControlFlowTok{function}\NormalTok{(}\AttributeTok{fondos =} \DecValTok{100}\NormalTok{, }\AttributeTok{apuesta =} \DecValTok{5}\NormalTok{)\{}
  \ControlFlowTok{while}\NormalTok{ (}\DecValTok{0}\SpecialCharTok{\textless{}}\NormalTok{fondos }\SpecialCharTok{\&}\NormalTok{ fondos}\SpecialCharTok{\textless{}}\DecValTok{200}\NormalTok{) \{}
    
\NormalTok{  \}}
  \FunctionTok{return}\NormalTok{(vec\_fondos) }\CommentTok{\# Devuelve un vector con el desarrollo de los fondos}
\NormalTok{\}}

\FunctionTok{plot}\NormalTok{(}\FunctionTok{ruina}\NormalTok{(), }\AttributeTok{type=}\StringTok{"l"}\NormalTok{, }\AttributeTok{col=}\StringTok{"blue"}\NormalTok{, }\AttributeTok{xlab=}\StringTok{"N° de juegos"}\NormalTok{, }\AttributeTok{ylab=}\StringTok{"Fondos"}\NormalTok{, }\AttributeTok{main=}\StringTok{"Evolución de los fondos (apuesta = 5)"}\NormalTok{)}
\FunctionTok{plot}\NormalTok{(}\FunctionTok{ruina}\NormalTok{(}\AttributeTok{apuesta =} \DecValTok{25}\NormalTok{), }\AttributeTok{type=}\StringTok{"l"}\NormalTok{, }\AttributeTok{col=}\StringTok{"blue"}\NormalTok{, }\AttributeTok{xlab=}\StringTok{"N° de juegos"}\NormalTok{, }\AttributeTok{ylab=}\StringTok{"Fondos"}\NormalTok{, }\AttributeTok{main=}\StringTok{"Evolución de los fondos (apuesta = 25)"}\NormalTok{)}
\FunctionTok{plot}\NormalTok{(}\FunctionTok{ruina}\NormalTok{(}\AttributeTok{apuesta =} \DecValTok{50}\NormalTok{), }\AttributeTok{type=}\StringTok{"l"}\NormalTok{, }\AttributeTok{col=}\StringTok{"blue"}\NormalTok{, }\AttributeTok{xlab=}\StringTok{"N° de juegos"}\NormalTok{, }\AttributeTok{ylab=}\StringTok{"Fondos"}\NormalTok{, }\AttributeTok{main=}\StringTok{"Evolución de los fondos (apuesta = 50)"}\NormalTok{)}
\end{Highlighting}
\end{Shaded}

~

A work by CC6104

~


\end{document}
